% !TEX TS-program = pdflatex
% !TEX encoding = UTF-8 Unicode

% This is a simple template for a LaTeX document using the "article" class.
% See "book", "report", "letter" for other types of document.

\documentclass[11pt]{article} % use larger type; default would be 10pt

\usepackage[utf8]{inputenc} % set input encoding (not needed with XeLaTeX)
\usepackage{alltt}
\usepackage{listings}
\usepackage{graphicx}
\usepackage{amsmath}
\usepackage{float}
\newcommand*\diff{\mathop{}\!\mathrm{d}}

%%% Examples of Article customizations
% These packages are optional, depending whether you want the features they provide.
% See the LaTeX Companion or other references for full information.

%%% PAGE DIMENSIONS
\usepackage{geometry} % to change the page dimensions
\geometry{a4paper} % or letterpaper (US) or a5paper or....
% \geometry{margin=2in} % for example, change the margins to 2 inches all round
% \geometry{landscape} % set up the page for landscape
%   read geometry.pdf for detailed page layout information

\usepackage{graphicx} % support the \includegraphics command and options

% \usepackage[parfill]{parskip} % Activate to begin paragraphs with an empty line rather than an indent

%%% PACKAGES
\usepackage{booktabs} % for much better looking tables
\usepackage{array} % for better arrays (eg matrices) in maths
\usepackage{paralist} % very flexible & customisable lists (eg. enumerate/itemize, etc.)
\usepackage{verbatim} % adds environment for commenting out blocks of text & for better verbatim
\usepackage{subfig} % make it possible to include more than one captioned figure/table in a single float
\usepackage{amsmath}
% These packages are all incorporated in the memoir class to one degree or another...

%%% HEADERS & FOOTERS
\usepackage{fancyhdr} % This should be set AFTER setting up the page geometry
\pagestyle{fancy} % options: empty , plain , fancy
\renewcommand{\headrulewidth}{0pt} % customise the layout...
\lhead{}\chead{}\rhead{}
\lfoot{}\cfoot{\thepage}\rfoot{}

%%% SECTION TITLE APPEARANCE
\usepackage{sectsty}
\allsectionsfont{\sffamily\mdseries\upshape} % (See the fntguide.pdf for font help)
% (This matches ConTeXt defaults)

%%% ToC (table of contents) APPEARANCE
\usepackage[nottoc,notlof,notlot]{tocbibind} % Put the bibliography in the ToC
\usepackage[titles,subfigure]{tocloft} % Alter the style of the Table of Contents
\renewcommand{\cftsecfont}{\rmfamily\mdseries\upshape}
\renewcommand{\cftsecpagefont}{\rmfamily\mdseries\upshape} % No bold!



\title{Numerical Analysis Homework 2}
\author{Margaret Dorsey}
%\date{} % Activate to display a given date or no date (if empty),
         % otherwise the current date is printed 

\begin{document}
\maketitle

\section*{Problem 1}
No, a degree 3 polynomial cannot intersect a degree 4 polynomial in exactly 5 points - let $f(x)$ be a degree $3$ polynomial, and $g(x)$ be degree 4. Any intersection of $f$ and $g$
must be a root of $f-g$ and $g-f$, which are degree 4 polynomials, and therefore may have at most $4$ real roots. 

\par It is, however, possible for $f$ and $g$ to intersect at exactly four points. Consider $f(x) = x^3$, and $g(x) = x^4 + 3x^3 - 12x^2 - 2x + 6$. Their difference,
$x^4+2x^3-12x^2 - 2x +6$, is plotted below and clearly has exactly $4$ distinct real roots. $f$ and $g$ can intersect only at those roots, and must intersect at them, so thus $f$ and $g$ intersect
exactly 4 times, as shown on the second plot. 

\includegraphics[width=0.5\textwidth]{plots/problem1plot1.png}
\includegraphics[width=0.5\textwidth]{plots/problem1plot2.png}
\section*{Problem 2}
Using the formula $$P_n(x) - \sum_{j=1}^n y_j \prod_{k=1, k \neq j}^n \frac{x-x_k}{x_j - x_k}$$
we obtain
	$$P_2(x) = \frac{(x-2)(x-4)}{3}\cdot 0 + \frac{(x-1)(x-4)}{-2}\cdot \ln2 + \frac{(x-1)(x-2)}{6} \cdot \ln4 $$
	$$P_2(x) =  \frac{-(x-1)(x-4)}{2}\cdot \ln2 + \frac{(x-1)(x-2)}{3} \cdot \ln2 $$
	$$P_2(x) =  \frac{-\ln2}{6}x^2 + \frac{3\ln2}{2}x -  \frac{4\ln2}{3} $$
The graph of this polynomial fit is shown below.
\begin{center}
\includegraphics[scale=.5]{plots/problem2plot1.png}
\end{center}
\par Using this approximation, we obtain $\ln 3 \approx P_2(3) =  \frac{-9\ln2}{6} + \frac{9\ln2}{2} -  \frac{4\ln2}{3} = \frac{5}{3} \ln 2 \approx 1.1552453$.
Using the formula
	$$|f(x) - P(x)|\leq \max_{[2,4]}\left| \frac{f^{(n+1)}( \xi)}{(n+1)!} \right| \cdot  \max_{[2,4]} \left| \prod_{i=0}^n(x-x_i) \right|$$
with $f(x) = \ln x$, $f^{(3)}(x) = \frac{2}{x^3}$, we obtain
	$$|f(x) - P(x)|\leq \max_{[2,4]}\left| \frac{1}{3 (\xi)^3} \right| \cdot \max_{[2,4]}  \left| x^3 - 7x^2 + 14x - 8 \right|$$
\par Because $f^{(3)}(\xi)$ is strictly decreasing and non-negative on our interval, we know the maximum on $[2,4]$ is $\frac{1}{24}$ at $2$. Using the derivative of $ x^3 - 7x^2 + 14x - 8$
with the quadratic formula, we find that the maximum absolute value on $[2,4]$ is $-\frac{2}{27}(10 + 7\sqrt7)$ at $x = \frac{7}{3}+\frac{\sqrt7}{3}$, or approximately $2.11261$.
\par Multiplying these values, we obtain $|f(x) - P(x)| \leq .0880255$. \\
\par Comparing our value of  $1.1552453$ to the actual value of $\ln 3 \approx 1.0986123$, we have $|f(x) - P_2(x)| = .056630 < .0880255$, as expected.
\par
\section*{Problem 3}
Using the formulas $B(0) = P_0$ and $B(1) = P_3$, we obtain $P_0 = (1,1)$ and $P_3 = (9,1)$. We then calculate $B'(t) = (x'(t),y'(t)) = (12t+6t^2,3t^2 - 1)$, and note that because in general $B'(t) = 3(1-t)^2(P_1 - P_0) + 6(1-t)t(P_2-P_1) + 3t^2(P_3 - P_2)$, $B'(0) = 3(P_1 - P_0)$ and $B'(1) = 3(P_3 - P_2)$. Thus we have $P_1 = (1,\frac{2}{3})$ and $P_2 = (3,\frac{1}{3})$. This gives us the complete set of control points for the curve, namely
$$\{(1,1),  (1,\frac{2}{3}),   (3,\frac{1}{3}), (9,1) \}$$
\section*{Problem 4}

We first note that 
$$\begin{array}{lclcl} A_1 & = &  \frac{x_2 - x}{x_2 - x_1} & = & 1-x \\
		        A_2 & = & \frac{x_3 - x}{x_3 - x_2} & = & 2-x \\
		        A_3 & = & \frac{x_4 - x}{x_4 - x_3} & = & 3-x 
\end{array}$$

$$\begin{array}{lclcl} B_1 & = &  \frac{x - x_1}{x_2 - x_1} & = & x - 0 \\
		        B_2 & = & \frac{x -x_2}{x_3 - x_2} & = & x - 1 \\
		        B_3 & = & \frac{x - x_3}{x_4 - x_3} & = & x - 2 
\end{array}$$

$$\begin{array}{lclcl} C_1 & = &  \frac{(x_2 - x_1)^2}{6}[A_1^3 - A_1] & = & \frac{1}{6}(-x^3 + 3x^2 - 2x) \\
		        C_2 & =&  \frac{(x_3 - x_2)^2}{6}[A_2^3 - A_2] & = &\frac{1}{6}(-x^3 + 6x^2 - 11x + 6) \\
		        C_3 & =&  \frac{(x_4 - x_3)^2}{6}[A_3^3 - A_3] & = & \frac{1}{6}(-x^3 + 9x^2 - 26x + 24) 
\end{array}$$

$$\begin{array}{lclcl} D_1 & = &  \frac{(x_2 - x_1)^2}{6}[B_1^3 - B_1] & = & \frac{1}{6}(x^3 - x) \\
		        D_2 & =&  \frac{(x_3 - x_2)^2}{6}[B_2^3 - B_2] & = &\frac{1}{6}(x^3 - 3x^2 + 2x) \\
		        D_3 & =&  \frac{(x_4 - x_3)^2}{6}[B_3^3 - B_3] & = & \frac{1}{6}(x^3 - 6x^2 + 11x - 6) 
\end{array}$$

%%%%%%%%%%%%%%%%%%%%%%%%%%%%%%%%%%%%%%%%%%%%%%%%%%%
\subsection*{Natural Endpoint Condition} 
$$\begin{bmatrix} 1 & 0 & 0 & 0 \\ x_2 - x_1 & 2(x_3 - x_1) & x_3 - x_2 & 0 
		 \\ 0 & x_3 - x_2 & 2(x_4 - x_2) & x_4 - x_3 \\ 0 & 0 & 0 & 1
\end{bmatrix}
\begin{bmatrix} y_1'' \\ y_2'' \\ y_3'' \\ y_4'' \end{bmatrix} = 
\begin{bmatrix} 0 \\ 6\left( \frac{y_3 - y_2}{x_3 - x_2} - \frac{y_2-y_1}{x_2 - x_1} \right) \\ 
			 6\left( \frac{y_4 - y_3}{x_4 - x_3} - \frac{y_3-y_2}{x_3 - x_2} \right) \\ 0\end{bmatrix}$$

$$\begin{bmatrix} 1 & 0 & 0 & 0 \\ 1 - 0 & 2(2 -0) &2 -1 & 0 
		 \\ 0 & 2 - 1 & 2(3 -1) & 3 - 2 \\ 0 & 0 & 0 & 1
\end{bmatrix}
\begin{bmatrix} y_1'' \\ y_2'' \\ y_3'' \\ y_4'' \end{bmatrix} = 
\begin{bmatrix} 0 \\  6\left( \frac{2 - 5}{2 - 1} - \frac{5-3}{1 - 0} \right) \\ 
			 6\left( \frac{1 - 2}{3 - 2} - \frac{2-5}{2 - 1} \right) \\ 0\end{bmatrix}$$

$$\begin{bmatrix} 1 & 0 & 0 & 0 \\ 1 & 4 &1 & 0 
		 \\ 0 & 1 & 4 & 1 \\ 0 & 0 & 0 & 1
\end{bmatrix}
\begin{bmatrix} y_1'' \\ y_2'' \\ y_3'' \\ y_4'' \end{bmatrix} = 
\begin{bmatrix} 0 \\ -30 \\ 12 \\ 0\end{bmatrix}$$

$$
\begin{bmatrix} y_1'' \\ y_2'' \\ y_3'' \\ y_4'' \end{bmatrix} = 
\begin{bmatrix} 1 & 0 & 0 & 0 \\ -\frac{4}{15} & \frac{4}{15} & -\frac{1}{15} & \frac{1}{15} 
		 \\  \frac{1}{15} & -\frac{1}{15} & \frac{4}{15} & -\frac{4}{15} \\ 0 & 0 &0 & 1
\end{bmatrix}
\begin{bmatrix} 0 \\ -30 \\ 12 \\ 0\end{bmatrix}$$

$$
\begin{bmatrix} y_1'' \\ y_2'' \\ y_3'' \\ y_4'' \end{bmatrix} = 
\begin{bmatrix} 0 \\ -\frac{44}{5} \\\frac{26}{5} \\ 0\end{bmatrix}$$

Using these derived values of $y_i''$ with our $A$,$B$,$C$, and $D$ equations above, we have

$$\begin{array}{lclcl} y^{(cubic)}_1 & = &  y_1A_1 + y_2B_1 + y_1''C_1 + y_2'' D_1 & = 
							& -\frac{22}{15}x^3 + \frac{52}{15}x + 3\\
		        y^{(cubic)}_2  & = & y_2A_2 + y_3B_2 + y_2''C_2 + y_3'' D_2 & = 
							&  \frac{7}{3}x^3 - \frac{57}{5}x^2 + \frac{223}{15}x - \frac{4}{5}\\
		        y^{(cubic)}_3  & = & y_3A_3 + y_4B_3 + y_3''C_3 + y_4'' D_3 & =
							& -\frac{13}{15}x^3 + \frac{39}{5}x^2 - \frac{353}{15}x + \frac{124}{5}
\end{array}$$

\begin{center}
\includegraphics[scale=.5]{plots/problem4plot1.png}
\end{center}
%%%%%%%%%%%%%%%%%%%%%%%%%%%%%%%%%%%%%%%%%%%%%%%%%%%%
\subsection*{Curvature-Adjusted Endpoint Condition}
$$\begin{bmatrix} 1 & 0 & 0 & 0 \\ x_2 - x_1 & 2(x_3 - x_1) & x_3 - x_2 & 0 
		 \\ 0 & x_3 - x_2 & 2(x_4 - x_2) & x_4 - x_3 \\ 0 & 0 & 0 & 1
\end{bmatrix}
\begin{bmatrix} y_1'' \\ y_2'' \\ y_3'' \\ y_4'' \end{bmatrix} = 
\begin{bmatrix} k_1 \\ 6\left( \frac{y_3 - y_2}{x_3 - x_2} - \frac{y_2-y_1}{x_2 - x_1} \right) \\ 
			 6\left( \frac{y_4 - y_3}{x_4 - x_3} - \frac{y_3-y_2}{x_3 - x_2} \right) \\ k_n\end{bmatrix}$$

$$\begin{bmatrix} 1 & 0 & 0 & 0 \\ 1 - 0 & 2(2 -0) &2 -1 & 0 
		 \\ 0 & 2 - 1 & 2(3 -1) & 3 - 2 \\ 0 & 0 & 0 & 1
\end{bmatrix}
\begin{bmatrix} y_1'' \\ y_2'' \\ y_3'' \\ y_4'' \end{bmatrix} = 
\begin{bmatrix} k_n \\  6\left( \frac{2 - 5}{2 - 1} - \frac{5-3}{1 - 0} \right) \\ 
			 6\left( \frac{1 - 2}{3 - 2} - \frac{2-5}{2 - 1} \right) \\ k_n\end{bmatrix}$$

$$\begin{bmatrix} 1 & 0 & 0 & 0 \\ 1 & 4 &1 & 0 
		 \\ 0 & 1 & 4 & 1 \\ 0 & 0 & 0 & 1
\end{bmatrix}
\begin{bmatrix} y_1'' \\ y_2'' \\ y_3'' \\ y_4'' \end{bmatrix} = 
\begin{bmatrix} k_1 \\ -30 \\ 12 \\ k_n\end{bmatrix}$$

Choosing $k_1 = 100$, $k_2 = -100$, we have 

$$
\begin{bmatrix} y_1'' \\ y_2'' \\ y_3'' \\ y_4'' \end{bmatrix} = 
\begin{bmatrix} 1 & 0 & 0 & 0 \\ -\frac{4}{15} & \frac{4}{15} & -\frac{1}{15} & \frac{1}{15} 
		 \\  \frac{1}{15} & -\frac{1}{15} & \frac{4}{15} & -\frac{4}{15} \\ 0 & 0 &0 & 1
\end{bmatrix}
\begin{bmatrix} 100 \\ -30 \\ 12 \\ -100\end{bmatrix}$$

$$
\begin{bmatrix} y_1'' \\ y_2'' \\ y_3'' \\ y_4'' \end{bmatrix} = 
\begin{bmatrix} 100 \\ -\frac{632}{15} \\\frac{578}{15} \\ -100\end{bmatrix}$$

$$\begin{array}{lclcl} y^{(cubic)}_1 & = & -\frac{1066}{45} + 50x^2 - \frac{1094}{45}x + 3  \\
		        y^{(cubic)}_2  & = & \frac{121}{9}x^3 - \frac{307}{5}x^2 + \frac{3919}{45}x - \frac{512}{15} \\
		        y^{(cubic)}_3  & = & -\frac{1039}{45} + \frac{789}{5}x^2 - \frac{15809}{45}x + \frac{3872}{15}
\end{array}$$

\begin{center}
\includegraphics[scale=.5]{plots/problem4plot2.png}
\end{center}
%%%%%%%%%%%%%%%%%%%%%%%%%%%%%%%%%%%%%%%%%%%%%%%%%%%%

\subsection*{Clamped Endpoint Condition}
$$\begin{bmatrix} 2 & 1 & 0 & 0 \\ x_2 - x_1 & 2(x_3 - x_1) & x_3 - x_2 & 0 
		 \\ 0 & x_3 - x_2 & 2(x_4 - x_2) & x_4 - x_3 \\ 0 & 0 & 1 & 2
\end{bmatrix}
\begin{bmatrix} y_1'' \\ y_2'' \\ y_3'' \\ y_4'' \end{bmatrix} = 
\begin{bmatrix} 6\left(  \frac{y_2-y_1}{x_2 - x_1} -c_n  \right)\\ 6\left( \frac{y_3 - y_2}{x_3 - x_2} - \frac{y_2-y_1}{x_2 - x_1} \right) \\ 
			 6\left( \frac{y_4 - y_3}{x_4 - x_3} - \frac{y_3-y_2}{x_3 - x_2} \right) \\  6\left( c_n - \frac{y_4 - y_3}{x_4 - x_3} \right)\end{bmatrix}$$

$$\begin{bmatrix} 2 & 1 & 0 & 0 \\ 1 - 0 & 2(2 -0) &2 -1 & 0 
		 \\ 0 & 2 - 1 & 2(3 -1) & 3 - 2 \\ 0 & 0 & 1 & 2
\end{bmatrix}
\begin{bmatrix} y_1'' \\ y_2'' \\ y_3'' \\ y_4'' \end{bmatrix} = 
\begin{bmatrix} 6\left(  \frac{5-3}{1-0} -c_n  \right)\\  6\left( \frac{2 - 5}{2 - 1} - \frac{5-3}{1 - 0} \right) \\ 
			 6\left( \frac{1 - 2}{3 - 2} - \frac{2-5}{2 - 1} \right) \\  6\left( c_n -  \frac{1 - 2}{3 - 2} \right)\end{bmatrix}$$

$$\begin{bmatrix} 2 & 1 & 0 & 0 \\ 1 & 4 &1 & 0 
		 \\ 0 & 1 & 4 & 1 \\ 0 & 0 & 1 & 2
\end{bmatrix}
\begin{bmatrix} y_1'' \\ y_2'' \\ y_3'' \\ y_4'' \end{bmatrix} = 
\begin{bmatrix} 12-6c_n \\ -30 \\ 12 \\ 6(c_n +1) \end{bmatrix}$$

Choosing $c_n = 4$, we have

$$
\begin{bmatrix} y_1'' \\ y_2'' \\ y_3'' \\ y_4'' \end{bmatrix} = 
\frac{1}{45}\begin{bmatrix}26 & -7 & 2 & -1 \\ -7 & 14 & -4 & 2 
		 \\  2 & -4 & 14 & -7 \\ -1 & 2 & -7 & 26
\end{bmatrix}
\begin{bmatrix} -12 \\ -30 \\ 12 \\ 30\end{bmatrix}$$

$$
\begin{bmatrix} y_1'' \\ y_2'' \\ y_3'' \\ y_4'' \end{bmatrix} = 
\begin{bmatrix} -\frac{12}{5} \\ -\frac{36}{5} \\\frac{6}{5} \\ \frac{72}{5}\end{bmatrix}$$

$$\begin{array}{lclcl} y^{(cubic)}_1 & = & -\frac{4}{5}x^3 - \frac{6}{5}x^2 + 4x + 3  \\
		        y^{(cubic)}_2  & = & \frac{7}{5}x^3 - \frac{39}{5}x^2 + \frac{53}{5}x + \frac{4}{5} \\
		        y^{(cubic)}_3  & = & -\frac{11}{5}x^3 - \frac{63}{5}x^2 + \frac{101}{5}x - \frac{28}{5} 
\end{array}$$

\begin{center}
\includegraphics[scale=.5]{plots/problem4plot3.png}
\end{center}

%%%%%%%%%%%%%%%%%%%%%%%%%%%%%%%%%%%%%%%%%%%%%%%%%%%%


\subsection*{Parabolically Terminated Endpoint Condition}

$$\begin{bmatrix} 1 & -1 & 0 & 0 \\ x_2 - x_1 & 2(x_3 - x_1) & x_3 - x_2 & 0 
		 \\ 0 & x_3 - x_2 & 2(x_4 - x_2) & x_4 - x_3 \\ 0 & 0 & -1 & 1
\end{bmatrix}
\begin{bmatrix} y_1'' \\ y_2'' \\ y_3'' \\ y_4'' \end{bmatrix} = 
\begin{bmatrix} 0 \\ 6\left( \frac{y_3 - y_2}{x_3 - x_2} - \frac{y_2-y_1}{x_2 - x_1} \right) \\ 
			 6\left( \frac{y_4 - y_3}{x_4 - x_3} - \frac{y_3-y_2}{x_3 - x_2} \right) \\ 0\end{bmatrix}$$


$$\begin{bmatrix} 1 & -1 & 0 & 0 \\ 1 & 4 &1 & 0 
		 \\ 0 & 1 & 4 & 1 \\ 0 & 0 & -1 & 1
\end{bmatrix}
\begin{bmatrix} y_1'' \\ y_2'' \\ y_3'' \\ y_4'' \end{bmatrix} = 
\begin{bmatrix} 0 \\ -30 \\ 12 \\ 0\end{bmatrix}$$

$$
\begin{bmatrix} y_1'' \\ y_2'' \\ y_3'' \\ y_4'' \end{bmatrix} = 
\frac{1}{24}\begin{bmatrix}19 & 5 & -1 & 1 \\ -5 & 5 & -1 & 1 
		 \\  1 & -1 & 5 & -5 \\ 1 & -1 & 5 & 19
\end{bmatrix}
\begin{bmatrix} 0 \\ -30 \\ 12 \\ 0\end{bmatrix}$$

$$
\begin{bmatrix} y_1'' \\ y_2'' \\ y_3'' \\ y_4'' \end{bmatrix} = 
\begin{bmatrix} -\frac{27}{4} \\ -\frac{27}{4} \\\frac{15}{4} \\ \frac{15}{4}\end{bmatrix}$$

$$\begin{array}{lclcl} y^{(cubic)}_1 & = &  - \frac{27}{8}x^2 + \frac{43}{8} x + 3  \\
		        y^{(cubic)}_2  & = & \frac{13}{8}x^3 - \frac{63}{8}x^2 + \frac{37}{4}x + 2 \\
		        y^{(cubic)}_3  & = & \frac{15}{8}x^2 - \frac{83}{8}x - \frac{61}{4} 
\end{array}$$

\begin{center}
\includegraphics[scale=.5]{plots/problem4plot4.png}
\end{center}



%%%%%%%%%%%%%%%%%%%%%%%%%%%%%%%%%%%%%%%%%%%%%%%%%%%%


\subsection*{Not-a-Knot Endpoint Condition}

$$\begin{bmatrix} x_3 - x_2 & -(x_3- x_1) & x_2 - x_1 & 0 \\ x_2 - x_1 & 2(x_3 - x_1) & x_3 - x_2 & 0 
		 \\ 0 & x_3 - x_2 & 2(x_4 - x_2) & x_4 - x_3 \\ 0 & x_4 - x_3 & -(x_4 - x_2) & x_3 - x_2
\end{bmatrix}
\begin{bmatrix} y_1'' \\ y_2'' \\ y_3'' \\ y_4'' \end{bmatrix} = 
\begin{bmatrix} 0 \\ 6\left( \frac{y_3 - y_2}{x_3 - x_2} - \frac{y_2-y_1}{x_2 - x_1} \right) \\ 
			 6\left( \frac{y_4 - y_3}{x_4 - x_3} - \frac{y_3-y_2}{x_3 - x_2} \right) \\ 0\end{bmatrix}$$

$$\begin{bmatrix} -3 & 1 & 2 & 0 \\ 1 - 0 & 2(2 -0) &2 -1 & 0 
		 \\ 0 & 2 - 1 & 2(3 -1) & 3 - 2 \\ 0 &  -1 & 4 & -3
\end{bmatrix}
\begin{bmatrix} y_1'' \\ y_2'' \\ y_3'' \\ y_4'' \end{bmatrix} = 
\begin{bmatrix} 0 \\  6\left( \frac{2 - 5}{2 - 1} - \frac{5-3}{1 - 0} \right) \\ 
			 6\left( \frac{1 - 2}{3 - 2} - \frac{2-5}{2 - 1} \right) \\ 0\end{bmatrix}$$

$$\begin{bmatrix} -3 & 1 & 2 & 0 \\ 1 & 4 &1 & 0 
		 \\ 0 & 1 & 4 & 1 \\  0 &  -1 & 4 & -3
\end{bmatrix}
\begin{bmatrix} y_1'' \\ y_2'' \\ y_3'' \\ y_4'' \end{bmatrix} = 
\begin{bmatrix} 0 \\ -30 \\ 12 \\ 0\end{bmatrix}$$

$$
\begin{bmatrix} y_1'' \\ y_2'' \\ y_3'' \\ y_4'' \end{bmatrix} = 
\frac{1}{198}\begin{bmatrix}-62 & 12 & 21 & 7 \\ 16 & 48 & -15 & -5 
		 \\  -2 & -6 & 39 & 13 \\ -8 & -24 & 57 & -47
\end{bmatrix}
\begin{bmatrix} 0 \\ -30 \\ 12 \\ 0\end{bmatrix}$$

$$
\begin{bmatrix} y_1'' \\ y_2'' \\ y_3'' \\ y_4'' \end{bmatrix} = 
\begin{bmatrix} -\frac{6}{11} \\ -\frac{90}{11} \\\frac{36}{11} \\ \frac{78}{11}\end{bmatrix}$$

$$\begin{array}{lclcl} y^{(cubic)}_1 & = &  -\frac{14}{11}x^3 - \frac{3}{11}x^2 + \frac{39}{11} x + 3  \\
		        y^{(cubic)}_2  & = & \frac{21}{11}x^3 - \frac{108}{11}x^2 + \frac{144}{11}x - \frac{2}{11} \\
		        y^{(cubic)}_3  & = & \frac{7}{11}x^3 - \frac{24}{11}x^2 - \frac{24}{11}x +10 
\end{array}$$

\begin{center}
\includegraphics[scale=.5]{plots/problem4plot5.png}
\end{center}

%%%%%%%%%%%%%%%%%%%%%%%%%%%%%%%%%%%%%%%%%%%%%%%%%%%%


\section*{Problem 5}
\subsection*{Raw Output}
\begin{lstlisting}

5 DATA POINTS 

Initial Points: 

(-5.000000,0.038462) (-2.500000,0.137931) (0.000000,1.000000) 
(2.500000,0.137931) (5.000000,0.038462) 

x:		-3.750000
lagrange:	-0.162550
cubic:		-0.039693
curveadj:	-0.042043
actual:		0.066390

x:		-1.250000
lagrange:	0.629062
cubic:		0.666659
curveadj:	0.667129
actual:		0.390244

x:		1.250000
lagrange:	0.939904
cubic:		0.666659
curveadj:	0.667129
actual:		0.390244

x:		3.750000
lagrange:	-1.716761
cubic:		-0.039693
curveadj:	-0.042043
actual:		0.066390




10 DATA POINTS 

Initial Points: 

(-5.000000,0.038462) (-3.888889,0.062021) (-2.777778,0.114731) 
(-1.666667,0.264706) (-0.555556,0.764151) (0.555556,0.764151) 
(1.666667,0.264706) (2.777778,0.114731) (3.888889,0.062021) 
(5.000000,0.038462) 

x:		-4.444444
lagrange:	-0.209264
cubic:		0.046749
curveadj:	0.046273
actual:		0.048186

x:		-3.333333
lagrange:	0.160420
cubic:		0.087924
curveadj:	0.088051
actual:		0.082569

x:		-2.222222
lagrange:	0.115474
cubic:		0.147614
curveadj:	0.147580
actual:		0.168399

x:		-1.111111
lagrange:	0.523247
cubic:		0.515771
curveadj:	0.515781
actual:		0.447514

x:		0.000000
lagrange:	0.861538
cubic:		0.857126
curveadj:	0.857121
actual:		1.000000

x:		1.111111
lagrange:	0.523247
cubic:		0.515771
curveadj:	0.515781
actual:		0.447514

x:		2.222222
lagrange:	0.115474
cubic:		0.147614
curveadj:	0.147580
actual:		0.168399

x:		3.333333
lagrange:	0.160420
cubic:		0.087924
curveadj:	0.088051
actual:		0.082569

x:		4.444444
lagrange:	-0.209264
cubic:		0.046749
curveadj:	0.046273
actual:		0.048186




15 DATA POINTS 

Initial Points: 

(-5.000000,0.038462) (-4.285714,0.051633) (-3.571429,0.072700) 
(-2.857143,0.109131) (-2.142857,0.178832) (-1.428571,0.328859) 
(-0.714286,0.662162) (0.000000,1.000000) (0.714286,0.662162) 
(1.428571,0.328859) (2.142857,0.178832) (2.857143,0.109131) 
(3.571429,0.072700) (4.285714,0.051633) (5.000000,0.038462) 

x:		-4.642857
lagrange:	2.668349
cubic:		0.044513
curveadj:	0.044317
actual:		0.044334

x:		-3.928571
lagrange:	-0.301858
cubic:		0.060808
curveadj:	0.060861
actual:		0.060851

x:		-3.214286
lagrange:	0.180694
cubic:		0.088163
curveadj:	0.088149
actual:		0.088249

x:		-2.500000
lagrange:	0.101181
cubic:		0.138115
curveadj:	0.138119
actual:		0.137931

x:		-1.785714
lagrange:	0.259506
cubic:		0.237467
curveadj:	0.237466
actual:		0.238733

x:		-1.071429
lagrange:	0.451153
cubic:		0.468041
curveadj:	0.468041
actual:		0.465558

x:		-0.357143
lagrange:	0.891166
cubic:		0.886911
curveadj:	0.886911
actual:		0.886878

x:		0.357143
lagrange:	0.904422
cubic:		0.886911
curveadj:	0.886911
actual:		0.886878

x:		1.071429
lagrange:	0.433077
cubic:		0.468041
curveadj:	0.468041
actual:		0.465558

x:		1.785714
lagrange:	0.293649
cubic:		0.237467
curveadj:	0.237466
actual:		0.238733

x:		2.500000
lagrange:	0.008506
cubic:		0.138115
curveadj:	0.138119
actual:		0.137931

x:		3.214286
lagrange:	0.569927
cubic:		0.088163
curveadj:	0.088149
actual:		0.088249

x:		3.928571
lagrange:	-3.285979
cubic:		0.060808
curveadj:	0.060861
actual:		0.060851

x:		4.642857
lagrange:	77.271383
cubic:		0.044513
curveadj:	0.044317
actual:		0.044334



 Chebyshev Approximation 5 Nodes 

x:		-3.847104
chebyshev:	-0.134899
actual:		0.063290

x:		-1.469463
chebyshev:	0.714472
actual:		0.316524

x:		1.469463
chebyshev:	0.714472
actual:		0.316524

x:		3.847104
chebyshev:	-0.134899
actual:		0.063290




 Chebyshev Approximation 10 Nodes 

x:		-4.696737
chebyshev:	0.032438
actual:		0.043366

x:		-3.995283
chebyshev:	0.074601
actual:		0.058954

x:		-2.902743
chebyshev:	0.077647
actual:		0.106090

x:		-1.526062
chebyshev:	0.381200
actual:		0.300403

x:		0.000000
chebyshev:	0.730822
actual:		1.000000

x:		1.526062
chebyshev:	0.381200
actual:		0.300403

x:		2.902743
chebyshev:	0.077647
actual:		0.106090

x:		3.995283
chebyshev:	0.074601
actual:		0.058954

x:		4.696737
chebyshev:	0.032438
actual:		0.043366




 Chebyshev Approximation 15 Nodes 

x:		-4.696737
chebyshev:	-1.346627
actual:		0.043366

x:		-3.995283
chebyshev:	1.079891
actual:		0.058954

x:		-2.902743
chebyshev:	1.866100
actual:		0.106090

x:		-1.526062
chebyshev:	-4.361026
actual:		0.300403

x:		0.000000
chebyshev:	4.266356
actual:		1.000000

x:		1.526062
chebyshev:	2.565579
actual:		0.300403

x:		2.902743
chebyshev:	-6.869444
actual:		0.106090

x:		3.995283
chebyshev:	5.599214
actual:		0.058954

x:		4.696737
chebyshev:	0.085492
actual:		0.043366

x:		0.000000
chebyshev:	4.266356
actual:		1.000000

x:		0.008421
chebyshev:	4.167831
actual:		0.999929

x:		0.324991
chebyshev:	-0.455012
actual:		0.904471

x:		-0.576289
chebyshev:	4.949485
actual:		0.750690

x:		0.324991
chebyshev:	-0.455012
actual:		0.904471



\end{lstlisting}

\subsection*{RMS Errors}
{\renewcommand{\arraystretch}{1.3} %<- modify cell height to give data breathing room
\begin{tabular}{| c | c c c c |}
\hline
n & lagrange & natural & curve adjusted & chebyshev\\
\hline
5 & .89780 & .04383 & .04421 & \\
10 & .02010 & .00341 & .00341 &\\
15 & 427.32136 & .000001&  .000001 &\\
\hline
\end{tabular}
\subsection*{Analysis}

As $n$ increases, the error in the Lagrange approximation explodes due to Runge's phenomenon, despite the error at the interior points improving at a rate consistent with
with the spline approximation - the magnitude of the error near the endpoints more than overcomes the reduction in error everywhere else. The splines behave consistently better
overall as $n$ increases due to their piecewise construction which allows them to avoid erratic endpoint behavior. 
\par Fixing the second derivatives of the splines marginally improves
accuracy near the endpoints, but makes no marked difference in overall error, at least at these small values of $n$. This makes sense - because splines are piecewise, fixing the endpoint
behavior will only greatly effect the points near the endpoints, and thus any improvement does not propagate well through the rest of the approximation, where the majority of the 
error data will be determined.
\par %Chebyshev goes here

\newpage
\section*{Problem 6}
\subsection*{Coefficient Table}
\textbf{Note:} The first $10$ values of $a_n$ are the coefficients for $10$ nodes, so I have not duplicated the data.
\begin{table}[H]
\begin{tabular}{| c |c c c |}
\hline
$a_n$ & Function 1 & Function 2 & Function 3 \\
\hline
$a_0$ & 0.220277& 0 & -0.125724\\
$a_1$ & 0 &0.449218 & 0\\
$a_2$ &0.575761 & 0& -0.531218\\
$a_3$ & 0 &0.776220 & 0\\
$a_4$ &0.631361 & 0& 0.636758\\
$a_5$ & 0 &-0.011739& 0\\
$a_6$ &-0.555377 & 0& 0.183205\\
$a_7$ & 0 & -0.396909& 0\\
$a_8$ &0.146591 & 0&  -0.253667\\
$a_9$ & 0 & 0.298239& 0\\

\hline
\end{tabular}
\begin{tabular}{| c |c c c |}
\hline
$a_n$ & Function 1 & Function 2 & Function 3 \\
\hline
$a_{10}$ &-0.020277 & 0 & 0.135478 \\
$a_{11}$ &0 & -0.198222& 0\\
$a_{12}$ &0.001767 & 0 & -0.069607\\
$a_{13}$ &0 & 0.151939& 0\\
$a_{14}$ &-0.000107 & 0 & 0.039138\\
$a_{15}$ &0 & -0.130133& 0\\
$a_{16}$ &0.000005 & 0 & -0.021989\\
$a_{17}$ &0 &0.119113 & 0\\
$a_{18}$ &0 & 0 & 0.010027 \\
$a_{19}$ &0 & -0.114332& 0\\
\hline
\end{tabular}
\end{table}

\subsection*{Plots}


\subsection*{Analysis}




\section*{Problem 7}
\subsection*{Data Tables}
%$$ \begin{array}{l l l}
%T_{10}(x) &= &512x^{10}-1280x^8+1120x^6-400x^4+50x^2-1\\
%T'_{10}(x) &= &5120x^9-10240x^7+6720x^5-1600x^3+100x \\
%T''_{10}(x) &= &46080x^8-71680x^6+33600x^4-4800x^2+100\\
%\int T_{10}(x) &= &\frac{512}{11}*x^{11}-\frac{1280}{9}x^9+160x^7-80x^5+\frac{50}{3}x^3-x
%\end{array}$$
{\renewcommand{\arraystretch}{1.1} %<- modify cell height to give data breathing room
\begin{table}[H]
\begin{minipage}{.5\linewidth}
\centering
\caption*{$T_{10}(x)$}
\begin{tabular}{| c | c | c | c |}
\hline 
$n$ & $a_{T'_{10}}$ & $a_{T''_{10}}$ & $a_{\int T_{10}}$ \\
\hline \hline
0 & 0 & 500 & 0\\
1 & 20 & 0 & 0\\
2 & 0& 960 & 0\\
3 & 20& 0& 0\\
4 & 0 & 840& 0\\
5 & 20 & 0 & 0\\
6 & 0 & 640 & 0\\
7 & 20 & 0 & 0\\
8 & 0 & 360 & 0\\
9 & 20 & 0 & $\frac{1}{18}$\\
10 & 0 & 0 & 0\\
11 &  0& 0& $\frac{1}{22}$\\
\hline
\end{tabular}
\end{minipage} 
%$$ \begin{array}{l l l}
%T_{15}(x) &= &16384x^{15}-61440x^{13}+92160x^{11}  -70400x^9+28800x^7-6048x^5+560x^3-15x\\
%T'_{15}(x) &= &245760x^{14}-798720x^{12}+1013760x^{10}-633600x^8+201600x^6-30240x^4+1680x^2-15\\
%T''_{15}(x) &= &3440640x^{13}-9584640x^{11}+10137600x^9-5068800x^7+1209600x^5-120960x^3+3360x \\
%\int T_{15}(x) &= &1024x^{16}-\frac{30720}{7}x^{14}+7680x^{12}-7040x^{10}+3600x^8-1008x^6+140x^4-\frac{15}{2}x^2
%\end{array}$$

\begin{minipage}{.5\linewidth}
\centering
\caption*{$T_{15}(x)$}
\begin{tabular}{| c | c | c | c |}
\hline 
$n$ & $a_{T'_{15}}$ & $a_{T''_{15}}$ & $a_{\int T_{15}}$ \\
\hline \hline
0 & 15 & 0 & 0\\
1 & 0 & 3360 & 0\\
2 & 30& 0 & 0\\
3 & 0& 3240& 0\\
4 & 30 & 0& 0\\
5 & 0 & 3000 & 0\\
6 & 30 & 0 & 0\\
7 & 0 & 2640 & 0\\
8 & 30 & 0 & 0\\
9 & 0 & 2160 & 0\\
10 & 30 & 0 & 0\\
11 &  0& 1560 & 0\\
12 &  30& 0& 0\\
13 &  0& 840& 0\\
14 &  30& 0& $\frac{1}{28}$\\
15 &  0& 0& 0\\
16 &  0& 0& $\frac{1}{32}$\\
\hline
\end{tabular}
\end{minipage}
\end{table}
%$$ \begin{array}{l l l}
%T_{20}(x) &= &524288x^{20}-2621440x^{18}+5570560x^{16}-6553600x^{14}+4659200x^{12}-2050048x^{10}+ \\
%		& &549120x^8-84480x^6+6600x^4-200x^2+1\\
%T'_{20}(x) &= &10485760x^{19}-47185920x^{17}+89128960x^{15}-91750400x^{13}+55910400x^{11}-20500480x^9+ \\
%		& &4392960x^7-506880x^5+26400x^3-400x\\
%T''_{20}(x) &= &199229440x^{18}-802160640x^{16}+1336934400x^{14}-1192755200x^{12}+615014400x^{10} \\
%		& &- 184504320x^8+30750720x^6-2534400x^4+79200x^2-400 \\
%\int T_{20}(x) &= & \frac{524288}{21}x^{21}-\frac{2621440}{19}x^{19}+327680x^{17}-\frac{1310720}{3}x^{15}+358400x^{13}-186368x^{11} + \\
%			& & \frac{183040}{3}x^9-\frac{84480}{7}x^7+1320x^5-\frac{200}{3}x^3+x
%\end{array}$$
\begin{table}[H]
\begin{minipage}{.5\linewidth}
\centering
\caption*{$T_{20}(x)$}
\begin{tabular}{| c | c | c | c |}
\hline 
$n$ & $a_{T'_{20}}$ & $a_{T''_{20}}$ & $a_{\int T_{20}}$ \\
\hline \hline
0 & 0 & 4000 & 0\\
1 & 40 & 0 & 0\\
2 & 0& 7920 & 0\\
3 & 40& 0& 0\\
4 & 0 & 7680& 0\\
5 & 40 & 0 & 0\\
6 & 0 & 7280 & 0\\
7 & 40 & 0 & 0\\
8 & 0 & 6720 & 0\\
9 & 40 & 0 & 0\\
10 & 0 & 6000 & 0\\
11 &  40& 0& 0\\
12 &  0& 5120& 0\\
13 &  40& 0& 0\\
14 &  0& 4080& 0\\
15 &  40& 0& 0\\
16 &  0& 2880& 0\\
17 &  40& 0& 0\\
18 &  0& 1520& 0\\
19 &  40& 0&  $\frac{1}{38}$\\
20 &  0& 0& 0\\
21 &  0& 0& $\frac{1}{42}$\\
\hline
\end{tabular}
\end{minipage}
\end{table}

\subsection*{Analysis}
The tables above list the coefficients $a_n$ of $T_n$ for all relevant $n$ for the first two derivatives and integral of each $T_n$. The patterns in the data can be explained by

$$T_n(x) = \frac{1}{2} \left( \frac{1}{n+1}T'_{n+1}(x) - \frac{1}{n-1}T'_{n-1}(x) \right)$$
which gives us 
$$\int T_n(x) = \frac{1}{2} \left(  \frac{1}{n+1}T_{n+1}(x) - \frac{1}{n-1}T_{n-1}(x) \right)$$
and
$$T_{n+1}(x) = 2xT_n(x) - T_{n-1}(x)$$
which gives
$$T'_n(x) = \frac{1}{1-x^2} \left( -nxT_n(x) + nT_{n-1}(x) \right)$$
which we can of course take the derivative of to obtain higher order derivatives of $T_n$.
\section*{Problem 8}
\subsection*{Lagrange Interpolation}
$L_8(x) = -10717.01390x^8 + 41523.43743x^7 - 66130.60764x^6 + 55973.20311x^5 - 27150.02639x^4 +7547.542702x^3 - 1112.916584x^2 + 66.38116670x + .302$\\
\includegraphics[scale=.5]{plots/problem7plot1.png}
\subsection*{Cubic Spline Interpolation}
$$
f(x) = \left\{
        \begin{array}{ll}
            .302 - .473x - 2.802x^3 &  x < 0.2 \\
            .12974+2.11093x-12.91926x^2+18.72996x^3 & x < 0.3 \\
 	.47410379-1.3326681x-1.44057x^2+5.975866x^3 & x < 0.4 \\
 	-2.430901+20.45487x-55.90942x^2+51.36657x^3 &  x < 0.5 \\
 	26.420189-152.651673x+290.3036659x^2-179.442153x^3 &  x < 0.6 \\
 	-37.6981566+167.940056x-244.015883x^2+117.40204x^3 & x < 0.7 \\
 	5.37168-16.644978x+19.67702x^2-8.1660098x^3 & x < 0.8 \\
 	1.257759-1.2177579x+.39299789x^2-.130999x^3 &  x \geq .8 \\
        \end{array}
    \right.
$$
\includegraphics[scale=.5]{plots/problem7plot2.png}

\subsection*{Analysis}
The cubic spline interpolation produces a much more convincing interpolation between the data points - the Lagrange polynomial suffers from very erratic behavior
at the endpoints (similar to Runge's phenomenon), and indeed $f(.1) = 0.251906420711722$ is a much more consistent value with the data than $L_8(.1) = 1.14113749$. 



\end{document}
