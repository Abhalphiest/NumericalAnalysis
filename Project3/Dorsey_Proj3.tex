% !TEX TS-program = pdflatex
% !TEX encoding = UTF-8 Unicode

% This is a simple template for a LaTeX document using the "article" class.
% See "book", "report", "letter" for other types of document.

\documentclass[11pt]{article} % use larger type; default would be 10pt

\usepackage[utf8]{inputenc} % set input encoding (not needed with XeLaTeX)

%%% Examples of Article customizations
% These packages are optional, depending whether you want the features they provide.
% See the LaTeX Companion or other references for full information.

%%% PAGE DIMENSIONS
\usepackage{geometry} % to change the page dimensions
\geometry{a4paper} % or letterpaper (US) or a5paper or....
% \geometry{margin=2in} % for example, change the margins to 2 inches all round
% \geometry{landscape} % set up the page for landscape
%   read geometry.pdf for detailed page layout information

\usepackage{graphicx} % support the \includegraphics command and options

% \usepackage[parfill]{parskip} % Activate to begin paragraphs with an empty line rather than an indent

%%% PACKAGES
\usepackage{booktabs} % for much better looking tables
\usepackage{array} % for better arrays (eg matrices) in maths
\usepackage{paralist} % very flexible & customisable lists (eg. enumerate/itemize, etc.)
\usepackage{verbatim} % adds environment for commenting out blocks of text & for better verbatim
\usepackage{subfig} % make it possible to include more than one captioned figure/table in a single float
% These packages are all incorporated in the memoir class to one degree or another...

%%% HEADERS & FOOTERS
\usepackage{fancyhdr} % This should be set AFTER setting up the page geometry
\pagestyle{fancy} % options: empty , plain , fancy
\renewcommand{\headrulewidth}{0pt} % customise the layout...
\lhead{}\chead{}\rhead{}
\lfoot{}\cfoot{\thepage}\rfoot{}

%%% SECTION TITLE APPEARANCE
\usepackage{sectsty}
\allsectionsfont{\sffamily\mdseries\upshape} % (See the fntguide.pdf for font help)
% (This matches ConTeXt defaults)

%%% ToC (table of contents) APPEARANCE
\usepackage[nottoc,notlof,notlot]{tocbibind} % Put the bibliography in the ToC
\usepackage[titles,subfigure]{tocloft} % Alter the style of the Table of Contents
\renewcommand{\cftsecfont}{\rmfamily\mdseries\upshape}
\renewcommand{\cftsecpagefont}{\rmfamily\mdseries\upshape} % No bold!

%%% END Article customizations

%%% The "real" document content comes below...

\title{Numerical Analysis Project 3}
\author{Margaret Dorsey}
%\date{} % Activate to display a given date or no date (if empty),
         % otherwise the current date is printed 


\newenvironment{claim}[1]{\par\noindent\underline{Claim:}\space#1}{}
\newenvironment{proof}[1]{\par\noindent\underline{Proof:}\space#1}{\hfill $\blacksquare$}

\newcommand{\pder}[2][]{\frac{\partial#1}{\partial#2}}

\begin{document}
\maketitle

\section*{Lane Emden Equations}

More data can be found in the corresponding .txt files in the outputs directory. \\

\subsection*{n = .5}

$\Xi = 2.753100$ \\
$-\left(\pder[\theta]{\xi}\right)_{\xi = \Xi} = 0.500242$\\

\begin{tabular}{| c | c c c c |}
\hline
$\xi$ & $\theta$ & $\hat{M}$ &  $\hat{I}$  & $\hat{\Omega}$ \\
\hline
0 & 1 & 0 & 0 & 0 \\
.5 & .958594 & .517034 & .013161 & 3.307733\\
1.0 & 0.837851 &  3.965218 & 0.052130 & 1.662054\\
1.5 & 0.646511 & 12.497775 & 0.115720 & 1.115660\\
2.0 & 0.402580 & 26.388160 & 0.199817 & 0.849346\\
2.5 & 0.132636 & 42.431575 & 0.292766 & 0.702564\\
2.7 & 0.026741 & 47.076509 & 0.322150 & 0.670429\\
\hline
\end{tabular}

\subsection*{n = 1}
$\Xi = 3.142100$ \\
$-\left(\pder[\theta]{\xi}\right)_{\xi = \Xi} =0.318430$\\

\begin{tabular}{| c | c c c c |}
\hline
$\xi$ & $\theta$ & $\hat{M}$ &  $\hat{I}$  & $\hat{\Omega}$ \\
\hline
0 & 1 & 0 & 0 & 0 \\
.5 & 0.958851 & 0.510625 & 0.010080 & 3.779649\\
1.0 & 0.841772 & 3.774032 &0.039630 & 1.906359\\
1.5 &0.664997 & 11.201527 &0.086815 & 1.288487\\
2.0 & 0.454649 & 21.885479 & 0.146994 & 0.991357\\
2.5 & 0.239389 & 32.689292 & 0.211071 & 0.829708\\
3.0 & 0.047040 & 39.095204 & 0.257711 & 0.754742\\
3.10 & 0.013 & 39.444576 & 0.260960 & 0.750543\\
\hline
\end{tabular}

\subsection*{n = 2}

$\Xi = 4.353100$ \\
$-\left(\pder[\theta]{\xi}\right)_{\xi = \Xi} =0.127300$\\

\begin{tabular}{| c | c c c c |}
\hline
$\xi$ & $\theta$ & $\hat{M}$ &  $\hat{I}$  & $\hat{\Omega}$ \\
\hline
0 & 1 & 0 & 0 & 0 \\
.5 & 0.959353 & 0.498253 & 0.005227 & 5.248850\\
1.0 & 0.848929 & 3.440920 &0.020269 & 2.666237\\
1.5 & 0.695367 & 9.277393 & 0.043458 & 1.823020\\
2.0 &0.529836 &16.404083 &0.071768 & 1.422937\\
2.5 & 0.374739 & 22.793235 &0.101328 & 1.204538\\
3.0 & 0.241824 &27.213646 & 0.127545 & 1.083075\\
3.5 & 0.133969 &29.506644 & 0.145944 & 1.022508\\
4.0 & 0.048840 &30.241656 &0.154024 & 1.001799\\
4.3 & 0.006811 &30.297908 &0.154830 & 1.000058\\
\hline
\end{tabular}

\subsection*{n = 3}

$\Xi = 6.897200$ \\
$-\left(\pder[\theta]{\xi}\right)_{\xi = \Xi} =0.042440$\\

\begin{tabular}{| c | c c c c |}
\hline
$\xi$ & $\theta$ & $\hat{M}$ &  $\hat{I}$  & $\hat{\Omega}$ \\
\hline
0 & 1 & 0 & 0 & 0 \\
.5 & 0.959839 & 0.486443 & 0.002072 &8.335976\\
1.0 & 0.855310 & 3.160498 &0.007936 & 4.262277\\
1.5 & 0.719502 & 7.914317 & 0.016735 & 2.941765\\
2.0 &0.582851 &13.143968 &0.027220 & 2.318066\\
2.5 & 0.461127 & 17.590476 &0.038177 & 1.973579\\
3.0 & 0.359227 &20.815551 & 0.048537 & 1.770052\\
3.5 & 0.276263 &22.913024 & 0.057505 & 1.647475\\
4.0 &0.209282 &24.161423 &0.064605 & 1.574866\\
4.5 &0.155069 &24.840997 &0.069681 & 1.533989\\
5.0 & 0.110900 & 25.171911  &0.072868&1.513033\\
5.5 & 0.074353 & 25.309843  &0.074544 &1.503789\\
6.0 &0.043794 &25.353617 &0.075200 & 1.500692\\
6.5 & 0.017914 &25.361601 &0.075344 & 1.500099\\
6.8 &0.004211 &25.361900 &0.075350 & 1.500077\\
\hline
\end{tabular}
\subsection*{Modelling Earth's Sun}

\section{White Dwarfs}
\subsection*{$\pder[V]{s}$}
\subsection*{Integrations for Selected Values of $\theta(0)$}
%\begin{figure}
%\begin{tabular}{c c}
 % \includegraphics[width=0.5\textwidth]{problem1plot2.png} &   \includegraphics[width=0.5\textwidth]{problem1plot2.png} \\
%$\theta(0) = $ & $\theta(0) = $ \\[6pt]
%\includegraphics[width=0.5\textwidth]{problem1plot2.png} &   \includegraphics[width=0.5\textwidth]{problem1plot2.png} \\
%$\theta(0) = $ & $\theta(0) = $ \\[6pt]
% \includegraphics[width=0.5\textwidth]{problem1plot2.png} &   \includegraphics[width=0.5\textwidth]{problem1plot2.png}\\
%$\theta(0) = $ & $\theta(0) = $ \\[6pt]
% \includegraphics[width=0.5\textwidth]{problem1plot2.png} &  \includegraphics[width=0.5\textwidth]{problem1plot2.png} \\
%$\theta(0) = $ & $\theta(0) = $ \\[6pt]

%\end{tabular}
%\end{figure}

\subsection*{Dimensionless Mass}

\subsection*{Mass-Radius Relation}

\end{document}
