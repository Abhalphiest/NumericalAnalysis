% !TEX TS-program = pdflatex
% !TEX encoding = UTF-8 Unicode

% This is a simple template for a LaTeX document using the "article" class.
% See "book", "report", "letter" for other types of document.

\documentclass[11pt]{article} % use larger type; default would be 10pt

\usepackage[utf8]{inputenc} % set input encoding (not needed with XeLaTeX)

%%% Examples of Article customizations
% These packages are optional, depending whether you want the features they provide.
% See the LaTeX Companion or other references for full information.

%%% PAGE DIMENSIONS
\usepackage{geometry} % to change the page dimensions
\geometry{a4paper} % or letterpaper (US) or a5paper or....
% \geometry{margin=2in} % for example, change the margins to 2 inches all round
% \geometry{landscape} % set up the page for landscape
%   read geometry.pdf for detailed page layout information

\usepackage{graphicx} % support the \includegraphics command and options
\usepackage{alltt}
\usepackage{wrapfig}
% \usepackage[parfill]{parskip} % Activate to begin paragraphs with an empty line rather than an indent

%%% PACKAGES
\usepackage{booktabs} % for much better looking tables
\usepackage{array} % for better arrays (eg matrices) in maths
\usepackage{amssymb}
\usepackage{paralist} % very flexible & customisable lists (eg. enumerate/itemize, etc.)
\usepackage{verbatim} % adds environment for commenting out blocks of text & for better verbatim
\usepackage{subfig} % make it possible to include more than one captioned figure/table in a single float
% These packages are all incorporated in the memoir class to one degree or another...

%%% HEADERS & FOOTERS
\usepackage{fancyhdr} % This should be set AFTER setting up the page geometry
\pagestyle{fancy} % options: empty , plain , fancy
\renewcommand{\headrulewidth}{0pt} % customise the layout...
\lhead{}\chead{}\rhead{}
\lfoot{}\cfoot{\thepage}\rfoot{}

%%% SECTION TITLE APPEARANCE
\usepackage{sectsty}
\allsectionsfont{\sffamily\mdseries\upshape} % (See the fntguide.pdf for font help)
% (This matches ConTeXt defaults)

%%% ToC (table of contents) APPEARANCE
\usepackage[nottoc,notlof,notlot]{tocbibind} % Put the bibliography in the ToC
\usepackage[titles,subfigure]{tocloft} % Alter the style of the Table of Contents
\renewcommand{\cftsecfont}{\rmfamily\mdseries\upshape}
\renewcommand{\cftsecpagefont}{\rmfamily\mdseries\upshape} % No bold!

%%% END Article customizations

%%% The "real" document content comes below...

\title{Numerical Analysis Project 2}
\author{Margaret Dorsey}
%\date{} % Activate to display a given date or no date (if empty),
         % otherwise the current date is printed 

\newenvironment{claim}[1]{\par\noindent\underline{Claim:}\space#1}{}
\newenvironment{proof}[1]{\par\noindent\underline{Proof:}\space#1}{\hfill $\blacksquare$}

\begin{document}
\maketitle

\section*{Justification of Method}
	I elected to use cubic splines due to the discrete nature of the data set - because we have no guarantee of consistent, polynomial-like behavior at the data points, it is likely that a single interpolating polynomial would suffer from erratic behavior in between points to accommodate inconsistencies. Although Bezier Curves would likely provide a smooth, appealing interpolation of the data, because they do not necessarily agree with the data points, we would have less fine control over their behavior which is problematic for precise determination of ideal lighting. With  cubic splines, we can make adjustments solely through manipulation of the number, spacing, etc of data points without the need to also tweak control points to get expected behavior.
% will we need a root finding algorithm to recover the coefficients?
\par Additionally, the cubic spline method makes it relatively easy to compute each interpolating polynomial. For lagrange polynomials, we would need to use a root finding method to determine each coefficient of the polynomial in order to take the derivative, because we generally compute them iteratively, and do not explicitly generate the polynomial. We do happen to know $c_0$ ahead of time, though, because one of the given data points in the set is $0$.

\subsection*{Computation of Splines}

\subsection*{Plots of Output}
\subsubsection*{Data Set 1}
\subsubsection*{Data Set 2}
\subsubsection*{$1-x^2$}
\subsubsection*{$\sqrt{5-x^2}$}

\subsection*{RMS Error Table}
{\renewcommand{\arraystretch}{1.3} %<- modify cell height to give data breathing room
\begin{tabular}{| c | c |}
\hline
Interpolated Function & RMS Error \\
\hline
$- \cos(x - 0.2)$ &  \\
$1 - x^2$ & \\
$\sqrt{5-x^2}$ & \\
\hline
\end{tabular}

\subsection*{Error Bounds}

\subsection*{Accuracy of Derivative Values}
\end{document}


 





